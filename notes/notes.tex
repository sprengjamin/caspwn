\documentclass[onecolumn%
              ,superscriptaddress%
              ,aps%
              ,pra]{revtex4}

\usepackage{amsmath,amssymb}
\usepackage{braket}
\usepackage{graphicx}
\usepackage{bm} % bold math symbols

% comment
\newcommand{\comment}[1]{{\it \color{RoyalBlue} #1}}

% log1p
\DeclareMathOperator{\Log1p}{log1p}

\begin{document}

\title{Asymptotics of special functions}

%\author{UFRJ - U. Augusburg collaboration}
%\date{\today}

\begin{abstract}
We discuss the asymptotics of the angular functions and Bessel functions involved in the Fredholm calculations.
\end{abstract}

\maketitle
\section{TO DO:}
\begin{itemize}
\item profiling with numba
\item improve speed of frac(nu,x) for certain parameters using asymptotics
\item test I0(x) and I1(x), test corner case x=0. for array
\item argue which values the parameters take for given precision in the Casimir calculation and write test for these parameter ranges, perhaps visualize the range in which everything is tested
\item evaluation of legendres can maybe be relaxed if we stick with using asymptotics after $\ell=1000$.
\item test accuracy of recursion up to $\ell=1000$.
\item $\ell=1000$, $x\simeq 0.1$ gives relative error of $10^{-15}$ for Mie coefficients, investigate
\item small error for pitau, when $x$ is much larger than $\ell$.
\item find numerically stabler expressions for $A$ to $D$, cancellation error?!
\item implement recursion functions and use them to test asymptotics
\item replace np.exp by math.exp if no arrays are used, check that this is actually faster (learned at EuroScipy2018)
\end{itemize}
\section{Modified Bessel functions}

The modified Bessel functions satisfy the recurrence relations
\begin{align}
I_{\nu+1}(x) &= - \frac{2\nu}{x} I_\nu(x) + I_{\nu-1}(x)\,,\label{eq:rec_besseli} \\
K_{\nu+1}(x) &= + \frac{2\nu}{x} K_\nu(x) + K_{\nu-1}(x)\,.
\end{align}

\subsection{low integer order}
For low integer order, which is needed for the evaluation of the Legendre polynomials, we compute the Bessel functions using Miller's algorithm.

We start by setting $\hat{I}_n(x)=A I_n(x)\equiv 1$ which means that we rescale by a constant $A$ which has to be determined. The rescaled Bessel function obeys the same recurrence relation~\eqref{eq:rec_besseli}. The rescaled Bessel function of order $n-1$ can be found by making use of the continued fraction representation of the fraction of bessel functions $f(n,x)=I_n(x)/I_{n+1}(x)=\hat{I}_n(x)/\hat{I}_{n+1}(x)$. We thus have $\hat{I}_{n-1}(x)=f(n-1,x)$. Using the recurrence relation~\eqref{eq:rec_besseli} repeatedly, we can then find $\hat{I}_0(x)$ or $\hat{I}_1(x)$. Since $I_0(x)$ and $I_1(x)$ can be determined accurately by a Chebyshev series expansion, we can determine the constant $A$ and hence an accurate value for $I_n(x)$. In theory it should be enough to determine $A$ solely by using $I_0(x)$. However, in numerical tests, it turned out that if $n$ is odd, it is more accurate to use $I_1(x)$ instead. 


\subsection{high order}
We use the uniform Debye asymptotics presented in the NIST library. For $\nu\rightarrow\infty$,
\begin{align}
I_\nu(\nu z) &\simeq \frac{e^{\nu \eta}}{(2\pi\nu)^{1/2}(1+z^2)^{1/4}} \sum_{k=0}^\infty \frac{U_k(p)}{\nu^k}\,, \\
K_\nu(\nu z) &\simeq \left(\frac{\pi}{2\nu}\right)^{1/2}\frac{-e^{\nu \eta}}{(1+z^2)^{1/4}} \sum_{k=0}^\infty (-1)^k \frac{U_k(p)}{\nu^k}\,,
\end{align}
where
\begin{align}
\eta &= (1+z^2)^{1/2} + \log\frac{z}{1+(1+z^2)^{1/2}} = (1+z^2)^{1/2} - \mathrm{arcsinh}(1/z)\,, \\
p & = (1+z^2)^{-1/2}
\end{align}
and $U_k(p)$ are polynomials of degree $3k$, defined by the recurrence relation
\begin{equation}
U_{k+1}(p) = \frac{1}{2}p^2(1-p^2) U_k^\prime(p) + \frac{1}{8}\int_0^p (1-5t^2) U_k(t)\mathrm{d}t
\end{equation}
beginning with $U_0(p)=1$. For $k=1,2,3$,
\begin{equation}
\begin{aligned}
U_1(p) &= \frac{1}{24}(3p-5p^3)\,, \\
U_2(p) &= \frac{1}{1152}(81p^2-462p^4+385p^6)\,, \\
U_3(p) &= \frac{1}{414720}(30375p^3-369603p^5+765765p^7 - 425425 p^9)\,. \\
\end{aligned}
\end{equation}

We define the exponentially scaled modified Bessel functions intuitively as
\begin{align}
\tilde I_\nu(\nu z) = e^{-\nu \eta} I_\nu(\nu z)\,,\\
\tilde K_\nu(\nu z) = e^{\nu \eta} K_\nu(\nu z)\,.
\end{align}

For the exponentially scaled modified Bessel function of the second kind the recurrence relation reads
\begin{equation}
\tilde K_{\nu+1}(x) = + \frac{2\nu}{x}e^{\Psi_1} \tilde K_\nu(x) + e^{\Psi_2}\tilde K_{\nu-1}(x)
\end{equation}
with
\begin{align}
\Psi_1 &= \psi(\nu+1, x) - \psi(\nu, x)\,, \\
\Psi_2 &= \psi(\nu+1, x) - \psi(\nu-1, x)
\end{align}
and
\begin{align*}
\psi(\nu, x) &= \sqrt{\nu^2 + x^2} + \nu \log\left(x/t_\nu\right)\,, \\
t_\nu &= t_\nu(x) = \nu + \sqrt{\nu^2 + x^2}\,.
\end{align*}
Numerically more stable expressions for the exponentials are given by
\begin{align}
e^{\Psi_1} &= \frac{x}{t_{\nu+1}} \exp\left(\Delta_1 - \nu \Log1p\left(\frac{1+\Delta_1}{t_\nu}\right)\right)\,, \\
e^{\Psi_2} &= \frac{x^2}{t_{\nu-1}t_{\nu+1}}\exp\left(\Delta_2 - \nu \Log1p\left(\frac{2+\Delta_1}{t_{\nu-1}}\right)\right)
\end{align}
with
\begin{align*}
\Delta_1 &= \frac{2\nu+1}{\sqrt{(\nu+1)^2+x^2} + \sqrt{\nu^2 + x^2}}\,, \\
\Delta_2 &= \frac{4\nu}{\sqrt{(\nu+1)^2+x^2} + \sqrt{(\nu-1)^2 + x^2}}\,.
\end{align*}
Using the Wronskian the modified Bessel functions of the first kind can then be computed  in terms of
\begin{equation}
\tilde{I}_{\nu}(x) = \left[x \left(e^{-\Psi_1} \tilde{K}_{\nu+1}(x) + f_\nu(x) \tilde{K}_\nu(x)\right)\right]^{-1}\,.
\end{equation}

\section{Mie coefficients}

For a perfectly reflecting sphere the Mie coefficients are
\begin{align}
a_\ell(ix) &= (-1)^\ell \frac{\pi}{2}\frac{xI_{\ell-1/2} - \ell I_{\ell+1/2}(x)}{x K_{\ell-1/2}(x) + \ell K_{\ell+1/2}(x)}\,, \\
b_\ell(ix) &= (-1)^{\ell+1} \frac{\pi}{2} \frac{I_{\ell+1/2}(x)}{K_{\ell+1/2}(x)}\,.
\end{align}

Once again we define exponentially scaled Mie coefficients through the definitions of the exponentially scaled modified Bessel functions
\begin{align}
\tilde a_\ell(ix) &=  e^{-2\psi(\ell,x)} a_\ell(ix)\,,\\
\tilde b_\ell(ix) &=  e^{-2\psi(\ell,x)} b_\ell(ix)
\end{align}
with
\begin{equation}
\psi(\ell,x) = \sqrt{(\ell+1/2)^2 + x^2} + (\ell+1/2) \log\frac{x}{\ell+1/2+\sqrt{(\ell+1/2)^2+x^2}}\,.
\end{equation}
In terms of the expontentially scaled modified Bessel functions they are
\begin{align}
\tilde a_\ell(ix) &=  \frac{\pi}{2}\frac{x e^\Psi\tilde I_{\ell-1/2} - \ell \tilde I_{\ell+1/2}(x)}{x e^{-\Psi}\tilde K_{\ell-1/2}(x) + \ell \tilde K_{\ell+1/2}(x)}\,,\\
\tilde b_\ell(ix) &=  \frac{\pi}{2} \frac{\tilde I_{\ell+1/2}(x)}{\tilde K_{\ell+1/2}(x)}
\end{align}
where
\begin{equation}
\Psi(\ell, x) = \psi(\ell-1, x)- \psi(\ell, x)\,.
\end{equation}
In order to avoid numerical errors due to cancellation, $\Psi$ should be computed using the identities
\begin{equation}
\Delta = \sqrt{(\ell-1/2)^2 + x^2} - \sqrt{(\ell+1/2)^2 + x^2} = -\frac{2\ell}{\sqrt{(\ell+1/2)^2 + x^2} + \sqrt{(\ell-1/2)^2 + x^2}}
\end{equation}
and
\begin{equation}
(\ell-1/2)\log\frac{x}{t_-}-(\ell+1/2)\log\frac{x}{t_+} = 
\ell \log\left(1+\frac{1-\Delta}{t_-}\right)+\frac{1}{2}\log\frac{t_- t_+}{x^2}
\end{equation}
with
\begin{align}
t_- = \ell-1/2+\sqrt{(\ell-1/2)^2+x^2} \\
t_+ = \ell+1/2+\sqrt{(\ell+1/2)^2+x^2}
\end{align}
thus
\begin{equation}
e^\Psi = \frac{\sqrt{t_-t_+}}{x}\exp\left[\Delta + \ell \mathrm{log1p}(1-\Delta)/t_-\right]
\end{equation}
which gives values to machine precision.



\section{Legendre polynomials}

We make use of the methods dicussed in Hartmann's thesis.
However, we want to use exponentially scaled versions of these
in order to have an estimator for calulating the scattering amplitudes.

We employ the asymptotics only for large orders $\ell\geq 1000$.

\subsection{Small arguments}
For small arguments,
\begin{equation}
(\ell+1)\sinh x < 25\,,
\end{equation}
the Legendre polynomials can be evaluated using the asymptotic expansion
\begin{equation}\label{eq:Ple_low}
P_\ell(\cosh x) \simeq \sum_{n=0}^{12} \frac{f_n(x\nu)}{\nu^n}
\end{equation}
where $\nu=\ell+1/2$. The functions $f_n(x)$ vanish for odd values of $n$
and for even values of $n$ they are given by
\begin{equation}
\frac{f_n(x\nu)}{\nu^n} = (-1)^{n/2}\sum_{m=n/2}^{n} a_{n,m} x^m \nu^{m-n} I_m(x\nu)
\end{equation}
where the coefficients $a_{n,m}$ are given in Tab.~\ref{table:a_nm}. By using exponentially scaled modified Bessel functions $\tilde I_m(z) = I_m(z) e^{-z}$ the exponentially scaled Legendre polynomials
\begin{equation}
\tilde P_\ell(\cosh x) = P_\ell(\cosh x) e^{-(\ell+1/2)\vert x\vert}
\end{equation}
can be obtained from the same formula.

Formula~\eqref{eq:Ple_low} gives values at machine precision for $\ell > 1000$ which has been tested. Maybe the maximum value of $n$ can be lowered if we choose to stick with $\ell>1000$.

\begin{table}
{\def\arraystretch{1.5}\tabcolsep=6pt
\begin{tabular}{|c|c|c|c|c|c|c|c|}
\hline
$a_{n,m}$ & $n$ & $\dots$ &$\dots$&$\dots$&$\dots$&$\dots$& $n/2$ \\\hline
0 & $1$ & & & & & &\\\hline
2 & $\frac{1}{12}$ & $\frac{1}{8}$ & & & & &\\\hline
4 &$\frac{1}{160}$ & $\frac{7}{160}$ & $\frac{11}{384}$ & & & &\\\hline
6 & $\frac{61}{120960}$ & $\frac{671}{80640}$ & $\frac{101}{3584}$ & $\frac{173}{15360}$ & & &\\\hline
8 & $\frac{1261}{29030400}$ & $\frac{1261}{967680}$ & $\frac{217}{20480}$ & $\frac{90497}{3870720}$ & $\frac{22931}{3440640}$ & & \\\hline
10 & $\frac{79}{20275200}$ & $\frac{1501}{8110080}$ & $\frac{7034857}{2554675200}$ & $\frac{1676287}{113541120}$ & $\frac{10918993}{454164480}$ & $\frac{1319183}{247726080}$ & \\\hline
12 & $\frac{66643}{185980354560}$ & $\frac{1532789}{61993451520}$ & $\frac{3135577}{5367398400}$ & $\frac{72836747}{12651724800}$ & $\frac{2323237523}{101213798400}$ & $\frac{1396004969}{47233105920}$ & $\frac{233526463}{43599790080}$\\\hline
\end{tabular}
}
\caption{The coefficients $a_{n,m}$ for even values of $n$ are given for $m=n$ in the first column to $m=n/2$ in the last non-vanishing value to the right.}
\label{table:a_nm}
\end{table}

\subsection{Large arguments}
For large arguments,
\begin{equation}
(\ell+1)\sinh x > 25\,,
\end{equation}
we make use of the asymptotic expansion
\begin{equation}
P_\ell (\cosh x) \simeq \left(\frac{2}{\pi \sinh x}\right)^{1/2} \sum_{m=0}^{M-1} C_{\ell,m} \frac{\cosh[(m+\ell+1/2)x]}{\sinh^m x}\,.
\end{equation}
The coefficients $C_\ell,m$ are given in terms of the recurrence relation
\begin{equation}
C_{\ell,m+1} = \frac{(m+1/2)^2}{2(m+1)(\ell+m+3/2)} C_{\ell,m}
\end{equation}
with initial value
\begin{equation}
C_{\ell,0} = \frac{\Gamma(\ell+1)}{\Gamma(\ell+3/2)}\,.
\end{equation}

Numerically, we prefer to have an exponentially scaled version
\begin{equation}
\tilde P_\ell(\cosh x) = P_\ell(\cosh x) e^{-(\ell+1/2)\vert x\vert}
\end{equation} Furthermore the hyperbolic functions have to be rewritten in terms of exponentials, in order to avoid overflow.

We find
\begin{equation}
\tilde P_\ell(\cosh x) \simeq \left(\frac{1}{2\pi \sinh x}\right)^{1/2} \sum_{m=0}^{M-1} \tilde{C}_{\ell, m} \frac{1+e^{-(2m+2\ell+1)x}}{(1-e^{-2x})^m}\,.
\end{equation}
Note that in the recursion for the coefficients $\tilde{C}_{\ell,m}$, the factor of two has been cancelled by the ones of the hyperbolic functions,
\begin{equation}
\tilde C_{\ell,m+1} = \frac{(m+1/2)^2}{(m+1)(\ell+m+3/2)} \tilde C_{\ell,m}
\end{equation}
and they have the same initial value $\tilde C_{\ell,0} = C_{\ell,0}$.

In practice, it is faster and more accurate to compute $C_{\ell,0}$ via the asymptotic series
\begin{equation}
C_{\ell,0} \simeq \frac{1}{\sqrt{\ell}}\left(1-\frac{3}{8\,\ell}+\frac{25}{128\,\ell^2} - \frac{105}{1024\,\ell^3} + \frac{1659}{32768\,\ell^4} - \frac{6237}{262144\,\ell^5}+ \dots\right)
\end{equation}
which was obtained using mathematica.
For $\ell\geq 1000$ it yields values at machine precision.

\section{Angular functions}
It is convenient to absorb the factors $\frac{2\ell+1}{\ell(\ell+1)}$ into $\pi_\ell$ and $\tau_\ell$:
\begin{align}
p_\ell(x) &\equiv \frac{2\ell+1}{\ell(\ell+1)} \pi_\ell(x) = \left[P_{\ell-1}(x)-P_{\ell+1}(x)\right](1-x^2)^{-1} \\
t_\ell(x) &\equiv \frac{2\ell+1}{\ell(\ell+1)} \tau_\ell(x) = -x p_\ell(x) + (2\ell+1)P_\ell(x)
\end{align}
For small arguments $\ell<1000$ we use the recursion for $\pi_\ell$ and $\tau_\ell$ \cite{BH}
\begin{equation}
\begin{aligned}
\pi_\ell(x) &= \frac{2\ell-1}{\ell-1}x\pi_{\ell-1}(x) - \frac{\ell}{\ell-1}\pi_{\ell-2}(x) \\
\tau_\ell(x) &= \ell x \pi_\ell(x) - (\ell+1)\pi_{\ell-1}(x)
\end{aligned}
\end{equation}
beginning with $\pi_0 = 0$ and $\pi_1 = 1$ or, accordingly,
\begin{equation}
\begin{aligned}
p_\ell(x) &= \frac{2\ell+1}{\ell+1}\left[x p_{\ell-1}(x) - \frac{\ell-2}{2\ell-3}p_{\ell-2}(x)\right] \\
t_\ell(x) &= \ell x p_\ell(x) - \frac{(2\ell+1)(\ell-1)}{2\ell-1} p_{\ell-1}(x)
\end{aligned}
\end{equation}
beginning with $p_0 = 0$ and $p_1 = 3/2$.

Using the exponentially scaled Legendre polynomials we define the exponentially scaled functions $\tilde p_\ell$ and $\tilde t_\ell$ by means of
\begin{equation}
p_\ell(x) = \tilde p_\ell(x) e^{(\ell+1/2)\mathrm{arccosh}\vert x\vert}\,,\quad t_\ell(x) = \tilde t_\ell(x) e^{(\ell+1/2)\mathrm{arccosh}\vert x\vert}\,.
\end{equation}
Their definition is then given by
\begin{equation}
\begin{aligned}
\tilde p_\ell(x) & = \left[e^{-2\mathrm{arccosh}\vert x\vert}\tilde P_{\ell-1}(x)-\tilde P_{\ell+1}(x)\right](1-x^2)^{-1} \\
\tilde t_\ell(x) & = -x \tilde p_\ell(x) + (2\ell+1)e^{-\mathrm{arccosh}\vert x\vert}\tilde P_\ell(x)
\end{aligned}
\end{equation}
and the recursion relations are
\begin{equation}
\begin{aligned}
\tilde p_\ell(x) &= \frac{2\ell+1}{\ell+1}\left[xe^{-\mathrm{arccosh}\vert x\vert} \tilde p_{\ell-1}(x) - \frac{\ell-2}{2\ell-3}e^{-2\mathrm{arccosh}\vert x\vert}\tilde p_{\ell-2}(x)\right] \\
\tilde t_\ell(x) &= \ell x \tilde p_\ell(x) - \frac{(2\ell+1)(\ell-1)}{2\ell-1} e^{-\mathrm{arccosh}\vert x\vert} \tilde p_{\ell-1}(x)
\end{aligned}
\end{equation}
with the starting values $\tilde p_0 = 0$ and $\tilde p_1 = \frac{3}{2} e^{-3/2 \mathrm{arccosh}\vert x\vert}$

\subsection{Uniform expansion}

\begin{equation}
P^{-m}_\ell (\cosh x) = \frac{1}{\nu^m} \left(\frac{x}{\sinh x}\right)^{1/2} \sum_{k=0}^\infty c_k(x) \left(m+\frac{1}{2}\right)^{(k)} I_{m+k}(\nu x) \left(\frac{2x}{\nu}\right)^k
\end{equation}
with $\nu = \ell + \frac{1}{2}$ and the rising factorials are defined by
\begin{equation}
\left(m+\frac{1}{2}\right)^{(k)} = \frac{\Gamma(m+k+1/2)}{\Gamma(m+1/2)}\,.
\end{equation}
The coefficients $c_k^0(x)$ can be obtained from the series expansion
\begin{equation}
\left(2x\frac{\cosh x - \cosh\left(\sqrt{x^2-y}\right)}{y \sinh(x)}\right)^{\mu} = \sum_{k=0}^\infty c_k(x) y^k\,.
\end{equation}
The first coefficients are
\begin{equation}
\begin{aligned}
c_0 &= 1\,, \\
c_1 &= \mu\frac{1-  x \coth (x)}{4 x^2}\,, \\
c_2 &= \mu\frac{3 \mu +3 (\mu -1) x^2 \coth ^2(x)+4 x^2-6 (\mu +1) x \coth (x)+9}{96 x^4}\,.
\end{aligned}
\end{equation}
With the connection formula
\begin{equation}
P^m_\ell (z) = \frac{\Gamma(1+\ell+m)}{\Gamma(1+\ell-m)}P_\ell^{-m}(z)
\end{equation}
we find
\begin{equation}
\pi_\ell(\cosh x) = \frac{1}{\sinh x}P^1_\ell(\cosh x) \simeq \ell(\ell+1) \frac{1}{\nu} \left(\frac{x}{\sinh^3 x}\right)^{1/2} \sum_{k=0}^\infty c_k(x) \left(\frac{3}{2}\right)^{(k)} I_{k+1}(\nu x) \left(\frac{2x}{\nu}\right)^k
\end{equation}
with
\begin{equation}
\begin{aligned}
c_0 &= 1\,, \\
c_1 &= \frac{1-x \coth (x)}{8 x^2}\,, \\
c_2 &= \frac{8 x^2-3 x^2 \coth ^2(x)-18 x \coth (x)+21}{384 x^4}\,, \\
c_3 &= \frac{-3 x^3 \coth ^3(x)+40 x^2-15 x^2 \coth ^2(x)-81 x \coth (x)+99}{3072 x^6}\,.
\end{aligned}
\end{equation}

\section{Scattering amplitudes}
The scattering amplitudes are
\begin{equation}
\begin{aligned}
S_1(\Theta) &= \sum_{\ell=1}^\infty \frac{2\ell+1}{\ell(\ell+1)}
    \left[ a_\ell\pi_\ell\big(\cos(\Theta)\big) + b_\ell\tau_\ell\big(\cos(\Theta)\big)\right]\\
S_2(\Theta) &= \sum_{\ell=1}^\infty \frac{2\ell+1}{\ell(\ell+1)}
    \left[ a_\ell\tau_\ell\big(\cos(\Theta)\big) + b_\ell\pi_\ell\big(\cos(\Theta)\big)\right]\,,
\end{aligned}
\end{equation}
with the new definitions
\begin{equation}
\begin{aligned}
S_1(x, z) &= \sum_{\ell=1}^\infty 
    \left[ a_\ell(ix) p_\ell(-z) + b_\ell(ix) t_\ell(-z)\right]\\
S_2(x, z) &= \sum_{\ell=1}^\infty \left[ a_\ell(i x) t_\ell(-z) + b_\ell(i x) p_\ell(-z)\right]\,,
\end{aligned}
\end{equation}
with
\begin{equation}
p_\ell(-z) = (-1)^{\ell+1} p_\ell(z)\,,\quad t_\ell(-z) = (-1)^{\ell} t_\ell(z)
\end{equation}
we observe that $S_1$ is positive and $S_2$ negative.
Allowing the notation
\begin{equation}
\begin{aligned}
S_1(x, z) &= \sum_{\ell=1}^\infty 
    \left[ a_\ell(ix) p_\ell(z) + b_\ell(ix) t_\ell(z)\right]\\
S_2(x, z) &= -\sum_{\ell=1}^\infty \left[ a_\ell(i x) t_\ell(z) + b_\ell(i x) p_\ell(z)\right]\,,
\end{aligned}
\end{equation}
In terms of the exponentially scaled functions we find
\begin{equation}
\begin{aligned}
\tilde S_1(x, z) &= S_1(x, z) \exp(-2x\sqrt{(1+z)/2}) = \phantom{-}\sum_{\ell=1}^\infty 
    \left[ \tilde a_\ell(ix) \tilde p_\ell(z) +  \tilde b_\ell(ix) \tilde t_\ell(z)\right]e^{\chi(\ell,x,z)}\\
\tilde S_2(x, z) &= S_2(x, z) \exp(-2x\sqrt{(1+z)/2}) =-\sum_{\ell=1}^\infty \left[\tilde a_\ell(i x)\tilde t_\ell(z) +\tilde b_\ell(i x)\tilde p_\ell(z)\right]e^{\chi(\ell,x,z)}\,,
\end{aligned}
\end{equation}
with
\begin{equation}
\chi(\ell,x,z) = (\ell+1/2)\mathrm{arccosh}z + \sqrt{(2\ell+1)^2 + (2 x)^2} - (2\ell+1) \mathrm{arcsinh} \frac{\ell+1/2}{x} - 2 x \sqrt{(1+z)/2}\,.
\end{equation}
Since for large $x$ the scattering amplitudes grow exponentially with exponent $2 x \sqrt{(1+z)/2}$, in order to prevent overflow we need to substract that exponent from $\chi$. However, cancellation might appear, so we have to write it in a computer friendly way. First we write $\chi(\ell, x, z)$ as
\begin{equation}
\chi(\ell,x,z) = x\left\lbrace2\left[\sqrt{y^2+1} - \sqrt{(1+z)/2}\right] + y\left[\mathrm{arccosh}(z) - 2\mathrm{arcsinh} (y)\right]\right\rbrace
\end{equation}
with $\nu = \ell+1/2$, $y = \nu/x$ and rewrite
\begin{equation}
\sqrt{y^2 + 1} - \sqrt{(1+z)/2} = \Delta \frac{y+\sqrt{(z-1)/2}}{\sqrt{y^2+1} + \sqrt{(1+z)/2}}
\end{equation}
and
\begin{equation}
\mathrm{arccosh}(z) - 2\mathrm{arcsinh} (y) = \log\left(1+ \frac{t_\chi}{(y+\sqrt{y^2+1})^2}\right)
\end{equation}
where
\begin{equation}
t_\chi \equiv z+\sqrt{z^2-1}-(y+\sqrt{y^2+1})^2 = -  \frac{2(z-1)\Delta(\Delta + \sqrt{2(z-1)})}{\sqrt{z^2-1} + \sqrt{z^2-1+2(z-1)\Delta(\Delta + \sqrt{2(z-1)})}}
\end{equation}
where $\Delta = y-\sqrt{(z-1)/2}$. The calculation of $\Delta$ suffers potentially from a loss of accuracy due to the cancellation error when $y$ and $\sqrt{(z-1)/2}$ are very close values.

\section{Transformation between polarization basis}

The coefficients describing the transformation between the two polarization basis are given by
\begin{equation}
\begin{aligned}
A &=
\frac{\xi^4\cos(\varphi)-\big[\kappa_1\kappa_2+k_1k_2\cos(\varphi)\big]
\big[k_1k_2+\kappa_1\kappa_2\cos(\varphi)\big]}
{\xi^4-\big[\kappa_1\kappa_2+k_1k_2\cos(\varphi)\big]^2} \\
B & = -\frac{\xi^2 k_1 k_2\sin^2(\varphi)}
{\xi^4-\big[\kappa_1\kappa_2+k_1k_2\cos(\varphi)\big]^2} \\
C & = +\xi\frac{\kappa_2k_1^2+\kappa_1k_1k_2\cos(\varphi)}
{\xi^4-\big[\kappa_1\kappa_2+k_1k_2\cos(\varphi)\big]^2}\sin(\varphi) \\
D & = -\xi\frac{\kappa_1k_2^2+\kappa_2k_1k_2\cos(\varphi)}
{\xi^4-\big[\kappa_1\kappa_2+k_1k_2\cos(\varphi)\big]^2}\sin(\varphi)
\end{aligned}
\end{equation}
with $\varphi = \varphi_1 - \varphi_2$.

For $-\pi/2 \geq \varphi \geq \pi/2$ it is better to rewrite the denominators as
\begin{equation}
\mathcal{N} \equiv \xi^4-\big[\kappa_1\kappa_2+k_1k_2\cos(\varphi)\big]^2 = -\left[\xi^2(k_1^2 + k_2^2) + 2 \kappa_1 \kappa_2 k_1 k_2 \cos(\varphi) + k_1^2 k_2^2 \left(1 + \cos^2(\varphi)\right)\right]
\end{equation}
and the numerator of $A$ as
\begin{multline}
\mathcal{A}\equiv\xi^4\cos\varphi-\big[\kappa_1\kappa_2+k_1k_2\cos(\varphi)\big] \big[k_1 k_2+\kappa_1\kappa_2\cos(\varphi)\big]= \\-\left[\xi^2(k_1^2 + k_2^2)\cos(\varphi) + 2k_1^2 k_2^2 \cos(\varphi) + \kappa_1 \kappa_2 k_1 k_2\left(1 + \cos^2(\varphi) \right)\right]\,.
\end{multline}
When $\varphi$ is close to $\pi$ cancellation errors might occur and we need to rewrite the expressions as
\begin{equation}
\mathcal{N} = -\left[ 2 \delta - 4 s^2 \delta + 4 s^4 k_1^2 k_2^2 + \xi^2(k_1^2 + k_2^2)\right]
\end{equation}
and
\begin{equation}
\mathcal{A} = -\left[2\delta - 4 s^2 \delta + 4 s^4 \kappa_1 \kappa_2 k_1 k_2 -\xi^2 (k_1^2 + k_2^2)\cos\phi\right]
\end{equation}
with $\phi = \varphi-\pi$, $s = \sin(\phi/2)$, $c = \sin(\phi/2)$ and
\begin{equation}
\delta \equiv k_1^2 k_2^2 - \kappa_1\kappa_2 k_1 k_2 = - \frac{\xi^4 k_1^2 k_2^2 + \xi^2(k_1^2 + k_2^2)k_1^2 k_2^2}{k_1^2k_2^2 + \kappa_1 \kappa_2 k_1 k_2}\,.
\end{equation}


\begin{thebibliography}{99}
\bibitem{BH}
C. F. Bohren and D. R. Huffman,
\textit{Absorption and Scattering of Light by Small Particles}
(Wiley, New York, 1983), Chap. 4.

\end{thebibliography}

\end{document}
\grid
